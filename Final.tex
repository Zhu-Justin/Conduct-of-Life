\documentclass[12pt,letterpaper]{article}
\usepackage[utf8]{inputenc}
\usepackage[english]{babel}
\usepackage{amsmath}
\usepackage{amsfonts}
\usepackage{amssymb}
\usepackage{graphicx}
\usepackage[margin=1in]{geometry}
\usepackage{setspace}
\doublespacing
\author{Justin Zhu \\ Advised by Dr. Alexandre Victorovich Gontchar \\ Written for \textit{Conduct of Life}, taught by Professor Unger and Professor Puett}
\title{The Academy and its Aftermath: the Decay of the Liberal Arts Education and Western-Eastern Tradition}
\date{}

\newif\ifdraft
\pagestyle{empty}

\begin{document}
\clearpage\maketitle
\thispagestyle{empty}
\pagebreak
\tableofcontents

\pagebreak

\pagestyle{plain}
\setcounter{page}{1}


\section{Preface}
A fine Wednesday afternoon, sitting in the.   It's the six.  Unger bellows into the air.  It's somewhat unsettling that the very auditorium, a certain despondence hangs over the classroom.  This is a class with law students.  I peer over, glancing.  Beyond me, another student copies and pastes his resume onto his LinkedIn profile.  All seem impervious to the lectures of 

It had been a Harvard I had seeking.  Going about.

Harvard, the vestige of high American culture, high world culture seems to run hollow these days.  God is dead, proclaimed Nietzsche, and perhaps too is the liberal arts curriculum.

\section{A Time of Recruiting}
It doesn't come as a surprise.  

One is left to wonder what is the state of the liberal arts curriculum in this day and age, whether any of it exists at all.  The influx of computer science [insert citation] moreover, is illustrative of a larger issue that the liberal.

\section{Shakespeare and the Humanities}
We are left to ponder what then consistutes the formation of the self.  Individuals like Hamlet.  The problem with Hamlet is that he doesn't say what he means and doesn't mean what he says.  And I too, like Hamlet, have found myself at odds with myself, purporting that I embody a certain sophisticated individualism, while at the same time recruiting shamelessly for investment banking firms, hedge funds, and big tech companies like many.

\section{The Post-War Asian Culture}
Puett describes.  

But this is no counsel of despair. In 1773, Samuel Johnson visited the University of St. Andrews on his journey to the Western Isles of Scotland. St. Andrews is an ancient institution, one of the 25 or so oldest universities in the world, and yet 350 years in, it had evidently fallen on hard times. Fewer than 100 students remained, and one of its old colleges had been dissolved. "To see it pining in decay and struggling for life," Johnson noted, "fills the mind with mournful images and ineffectual wishes." He was under no illusion as to where the blame lay: "It is surely not without just reproach, that a nation, of which the commerce is hourly extending, and the wealth encreasing ... while its merchants or its nobles are raising palaces, suffers its universities to moulder into dust."

The humanities and the university do need defenders, and the way to defend the humanities is to practice them. Vast expanses of humanistic inquiry are still in need of scholars and scholarship. Whole fields remain untilled. We do not need to spend our time justifying our existence. All we need to do is put our hand to the plow. Scholarship has built institutions before and will do so again. Universities have declined and come to flourish once more. The humanities, which predate the university and may well survive it, will endure — even if there is no case to defend them.



\section{College}
Universities have become a big social club where.  Colleges are not about examining timeless truths.  Universities are not about building skillsets, at least in the humanities and social sciences.  Instead, universities are prioritizing credentials and big social clubs.  There is a vast impact on American society, the intellegensia, the foolish attitude of the white, working-class who did not go to college.  People don't take out their IQ scores.

The way people think about college is "He want to Harvard.  He must have been very smart."  This right-wing figure were over a conflict over a film.  This person said to Jeremy, you didn't even go to college.  People who go to college feel like they are more privileged.  The person who went to Yale is smarter than Junior college.  It is not universally true, it does not translate over to a level of success.  Innate ability does not translate to ability.

People can brag about where they went to school.  Most of us have institutions.  Church and work.  Their social fabric is created by them working with others.  School or church.  As churches declined and as schools become less of a binding committment, then social fabrics disappear.  Colleges have become a new social fabric.  



\section{The Academy}
The liberal arts curriculum is one often represented by multiple stages of human underpinnings.  The motivation, life as we know it is propelled into something deeper, more elemental.  Such are the purpose.  This paper is a dedication to a journal, manifesto of personal philosophies compiled and influenced by Nietzsche, Schopenhauer and other notable thinkers.  It is a tribute to the idea of human striving, of the mere ``essentials" of life, of tasting the bare essentials of life.

A major question has always been the purpose of the academy, of higher-education, the very liberal arts curriculum that an individual undertakes during the formidable years of college.  This education is formidable because it marks the supposedly complete maturation of the self, one where the individual will have explored deeper ideas of his own identity and personality.  The thinking goes that once a young individual is able to fully identify his/her own identity and personality, the individual will be able to identify .  The job market comes and goes, with new jobs created and destroyed all the time.

\section{Sun Tzu: The Art of War}
The Art of War is pretty important in that the individual is able to internalize something more 

\section{Youth and Self}

The youth is what is most impressive, characterizing the most important attributes of human surroundings.  Sometimes I woder who I am, the effusive underpinnings, lost in thought.

\section{Relationships}
The Confucius tradition necessitates a certain grounding of truth, where individuals are.  Confucianism is a broader set of ideas where individuals are compelled to look upon

\section{Individualism and Self-Reliance}
This is the individual and self-reliant perspective, where individuals are able to internalize and describe something more fundamental.  Individuals are able to internalize and describe the world, seeing a more versatile literary style.  People are more contrived to explore other pathways to human understanding.  My own personal manifesto.

\section{Work and Family}

\section{Romance}

\section{Principles}
Courage is the enabling virtue by which all other virtues fall apart. 

\section{Schopenhauer}
Jettisons everything, creating lead causality,  if you look  at the system of arts.  Causality is how we cognitively structure.  How we engage with the world artisitcally.  Defeat and offer another paradigm that is outside causality.  It's a fine thought.  Music is the highest form because it doesn't have the connection to the materiality of the world.  He doesn't represent everything.  This is similar to what conecpts are.  It's that some meaning are always connected in our mind, a string of meanings that lead us to create certain judgments.  Profoundly disconnected from the mateiral world of stuff.  A certain ontology of the world is offered ti the third world.  You can think aof everything you experience as having a liftime of its own.  We define life ontp a life of the function.  Things have their duration, such as a table tahat comes to life as a table.  Everything like the chairs.  This will is an ocean of energy that is self-contained, branching into multiple things.  In the worl dthat we inhabit we see many things have the duration that they do.  Everything is regent as we know it in the world.  Every one of us is a microcosm, everything one is a notion of ocur kind.  And in that sense, if we think about the self, we too exist, and how we live or do like these processes

What is the idea> The idea is that we do not to  It's only the ethical register where we wake up face to face with how the world fundamentally is.  Here we can do X but we are not going to do X, and so we inhibit that impulse.

I don't think there is a thing like Kant's second critique.  Kant is cloeset to the ancient thought, not every one of ius is different.  There's a way the world is primordially structured.  Plato and Aritotle is to see that eternal structure of the world, and get it from God's point of view.  Once you have that access, then you will be able to just do the right thing.  No Steve JObs, or Napolean.  They will just disregard the idea.  The narrative sself having the ontological weight.  It's a strange eclectic with Aristotle and Kant.

Nothing happens with this chari to sit on, in multiple spaces, there is no chair.  It was always there.  Everything we see in the world is an illusion, the smotley world of entities, things we could not be clinging to.  Look, you have to understand this hierarchy of being.  Here is where we offer a Buddhist take, where the best thing to do is to not do anything.

Kant's critique, Aristotle and Plato.  Philosopher is a mechanic of the soul, drawing out the talents of people, so that people would open up.  This is a good Buddhist thought, not as good as somebody with a high level of.  Systematic thinkers with eclecticism like Marx, Hegel.  Some aspects 

Schopenhauer is writing from a state of unhappiness, in contrast tot he Buddha, who is cominng from happiness.  Contrasting those dimensions in a very present way.  This juxtaposition between the two.  



\end{document}
