\documentclass[12pt,letterpaper]{article}
\usepackage[utf8]{inputenc}
\usepackage[english]{babel}
\usepackage{amsmath}
\usepackage{amsfonts}
\usepackage{amssymb}
\usepackage{graphicx}
\usepackage[margin=1in]{geometry}
\usepackage{setspace}
\doublespacing
\author{Justin Zhu \\ Advised by Dr. Alexandre Victorovich Gontchar \\ Written for \textit{Conduct of Life}, taught by Professor Unger and Professor Puett}
\title{SELF-RELIANCE IN THE DIGITAL AGE}
\date{MARCH 13, 2018}

\newif\ifdraft
\pagestyle{empty}

\begin{document}

\pagestyle{empty}
\begin{titlepage}
\pagestyle{empty}
\maketitle
\pagestyle{empty}
\pagebreak
\tableofcontents
\end{titlepage}
\pagebreak
\pagestyle{plain}
\setcounter{page}{1}
\section{Thesis}

In this paper, I will reconcile Emerson's vision of
self-reliance with certain challenges of conducting life in the digital age.
This digital age has witnessed fundamental changes
in human behavior and human interaction via the rise of
digital worlds and the projection of a digital self where
this digital self is divorced, and often deceiving, of the real, true self.    The rise
of social media platforms like Facebook and messaging services like Gmail and Slack have fundamentally altered human identities, ultimately distancing human beings from the self-reliance that Emerson advocated for.  Emerson's ideas are not a solution for conduct of life in the digital age -- in fact, I will identify in subsequent sections that certain components of Emerson's self-reliance are contradicting and incompatible with such realities of everyday life.  However, being said, certain Emersonian self-reliance are still relevant and, moreover, prescriptive for living better lives in this digital age by nature of encouraging individuals to become independent thinkers.  Thus, a reconciliation of Emersonian ideas and of communication in the digital age is probably the most productive viewpoint to be developed in this project concerning the conduct of life.


\end{document}
