\documentclass[12pt,letterpaper]{article}
\usepackage[utf8]{inputenc}
\usepackage[english]{babel}
\usepackage{amsmath}
\usepackage{amsfonts}
\usepackage{amssymb}
\usepackage{graphicx}
\usepackage[margin=1in]{geometry}
\usepackage{setspace}
\doublespacing
\author{Justin Zhu \\ Advised by Dr. Alexandre Victorovich Gontchar \\ Written for \textit{Conduct of Life}, taught by Professor Unger and Professor Puett}
\title{Modernism and the Career: An Investigation of Nonconformity and Self-Invention}

\date{}

\newif\ifdraft
\pagestyle{empty}

\begin{document}
\clearpage\maketitle
\thispagestyle{empty}
\pagebreak
\tableofcontents

\pagebreak

\pagestyle{plain}
\setcounter{page}{1}


\section{Thesis}
In this paper, I will proceed to describe
the greatest strength and the greatest weakness of what we
have discussed as the ``project of non-conformity and
self-invention."  To do so, I will first define the ideas of
non-conformity and self-invention in modernism, which allows us to explore specific qualities characterizing
non-conformity and self-invention, with these qualities
taking on extended meanings in context of a ``career" in
the modern world defined by ``modernism."

The focus of this essay will investigate the strengths and weaknesses of non-conformity and self-invention, and relating these strengths and weaknesses into one's career in the modern world.
\ifdraft
We will trace the development of the typical adult's career starting in university, identifying forces that contribute to non-conformity and self-invention along every stage of a typical adult's career in the modern world.
\fi
By extending the significance of these
strengths of non-conformity and self-invention into
notions of career, we better frame the challenges of the working man in the modern world.  This framework is necessary in helping us answer fundamental questions of how to identify worthwhile values in working society, how to best achieve those values in a working society, and how to maintain those values in an overarching project concerning the conduct of life.

\ifdraft

The working man is the brick that sustains larger society as a whole, and thereby understanding
the framework that governs challenges of non-conformity and self-invention in the modern world, we can and gain insight into certain societal attributes that possess more favorable attributes.

The technology for modern society is already there, but other parts of people's humanity leave us in question.  This is what we mean in broader society. 


preserving this brick in a most robust form yields longer-lasting benefits that enable society to better thrive beyond the constraints of the world at large.

such that this working man can experience certain strengths of nonconformity and self-invention will ultimately allow a proliferation of human desires that extend beyond the s

which has always been largely interpreted
by society as one's main role within the larger context of
society, how one best proceeds to work in one's career in
making the most out of non-conformity and self-invention.


By setting something like textwidth, we thereby obtain
something that can have large interpretative qualities.  It
is for these interpretation that we live our lives, moving
on undeterred in seizing the currents that make us happier
individuals.  For that reason, I think I've found a greater
outlet for happiness.  I know vim, and this is always a
good thing.

Another point I'd like to just point out is the cyclical
nature of religion.  It is extremely hard in my opinion to
ever construe something that has literary merit if we are
always bickering over what we think to be right or what we
think to be wrong.  Such discourses usually always lead us
into the wrong direction, one where we are unable to know
for ourselves what is write and what is wrong.  This is
quite problematic sometimes.

The journey of the writer is a solitary one, but I just
need to churn out something more that the New Yorker likes,
and I should be on my way to publishing more papers, and
expanding my intellectual horizons.
\fi

\section{A Definition of Modernism}
Unger defines modernism as a moment in the transformation of a Christian romantic idea of the self.  This definition carries several implications.  For one thing, an established Christian-romantic tradition represents an assumption of existing societal order, where certain principles like love and faith are established as prevailing values for the individual to strive for.  The very ``transformation" of a Christian romantic idea in modernism suggests that modernism is colored with new developments deviating from original, pure Christian romantic ideas, as these developments and events over the course of human history have created new paradigms that original definitions may have been unable to account for.  However, these new developments do not war against Christian romantic ideas of the self because such warring would not constitute a transformation, but provide an antithesis, when framing this relationship under Hegel's ``dialectic."

Using this definition of modernism, non-conformity and self-invention present a fundamental characterization of the individual never being content with the world, that is, according to Unger, "the man is never at home in the world."  To make this world more homely, individuals can transform the world by infusing more faith, hope, and love into this world, as such ideas are consistent with the values of the Christian-romantic tradition.  


The striving of these Christian-romantic values creates nonconformity in that the individual is unwilling to accept a certain social order that characterizes the current world, because social order is often at odds with a deeper encounter with the ideas of faith, hope, and love.  Moreover, because these ideas of faith, hope, and love are personalized to the individual experience, the manner in which individuals define themselves by the way they strive for these ideals forms the basis of self-invention in this world of modernism.  

\ifdraft

When understanding the role of Christian element in the world, we come to an formalization of our transformation over in the timeline of a career, where we make the world that we live in more ``homely" so that we can better appreciate.  
\fi

\section{Christian-Romantic Traditions and the Career of the Working Man}
We can identify the parallels between the journeys of the hero in early Romantic tradition with that of the working man embarking on a career in contemporaneous society.  Using this parallelization, we can better identify elements of non-conformity and self-invention characteristic of modernism for the working man in today's society.

In typical Christian-romantic tradition, the familiar protagonist is one of a young adventurer who tries to remove the specific obstacle to human happiness, usually one that stands in the way of personal happiness.  Through this journey of seeking personal happiness, the individual experiences greater understanding of the self, casting away a certain naivety of initial expectations for how the journey will unfold.  Despite the forces that clash with the individual's aspirations, the individual still retains a basic confidence and faith in morality and the self at the end.  In this way, the values of the individual are strengthened over the course of this journey.

A career is no less different in the sense that the career is a long journey where the hero, in this context the working man, experiences many obstacles that stand in the way of his own personal happiness.  A fulfilling career exhibits the ultimate narrative arc of the Christian-romantic tradition, where the working man is able to build upon his work and experiences to achieve a long-lasting happiness and celebration of personal values accumulated over many years of sustained and devoted effort.  A fulfilling career incorporates great challenges and obstacles that are not trivial, for these challenges ultimately provide the greatest impetus for strengthening an individual's value by the end of this journey.

In both the hero's quest within Christian-romantic tradition and the working man's career within contemporaneous society, elements of self-invention and non-conformity appear.  Self-invention exists in that the working man has the option to choose his career, and to make decisions that reflect deeper values over the course of this career.  By looking back upon this career, the working man is able to identify a sense of self that has been ``invented" by these decisions.  

Finally, the working man's career exhibits possibility for non-conformity by how the individual is able to fight against obstacles of an impersonal society that threaten the accomplishment of the individual's goal.  The creation of a fulfilling career demands not giving in to society's pressures until that vision of happiness is truly realized, to ultimately not conform.  Having established non-conformity and self-invention as elements prevalent in both the Christian-romantic tradition and contemporaneous society, we now proceed to analyze strengths and weaknesses within each of these values.  These strengths and weaknesses will help us identify the underpinnings of an optimal balance and the ultimate achievement of a fulfilling career in contemporaneous society.

\section{Relating the Strengths and Weaknesses of Nonconformity and Self-Invention}

The strengths of self-invention and non-conformity are largely attributed to its necessity in realizing the values of the Christian-romantic world that lead to a fulfilling career.  As stated in the previous section, a fulfilling career is one  predicated on removing obstacles and not conforming to the pressures of society.  This struggle against obstacles and the pressures of society ultimately leads to the triumph of the individual by removing these obstacles in achieving a truer, longer-lasting happiness.

Seeing that the strengths of nonconformity and self-invention are readily apparent in the path to a fulfilling career, it may be important to acknowledge that adhering too strongly to these values of nonconformity and self-invention can lead to self-destructive effects, thus indicating a certain weakness within nonconformity and self-invention.	


\ifdraft

 relationship between the strengths and weaknesses of nonconformity and self-convention.  

It's important to understand that 

Ideally, a full embrace of these values nonconformity and self-invention will allow an individual to quickly develop a fulfilling path to goodness.  It is this fulfillment that ultimately makes an individual more happy and free in understanding the world.  

These people, these individuals are sometimes divorced from broader pictures and broader understanding.  Such understanding is often peripheral, but not the main enconcompassing understanding which makes individuals truly happy and figuring.  These ideas are not one where we can divorce and isolate

One of these challenges in the Christian world is the inability to create broader challenges characterizing the world as whole.  These challenges are a facet to understanding the ways individuals themselves hold themselves to a higher standard 

The strengths within society is one where individuals are not necessarily here to understand the motivations that guide them, that propel them to greater heights.  Society's strengths are largely encompassing and represent a facet of life that is true to the individual.  During moments like these, the individual is truly isolated from the world at large, unresisting in the forces at play.  

understand more of the precarious understandings that define the individual and how such understandings are here for the 

 These values are contained within the individual's understanding of the world, the readily important devotions and characterizations that make the world his own.  Creativity and 

experience that unites the individual in his quest to make the world a better place.  

These obstacles are enough to make the hero more appreciative of other qualities that define the

As Unger explains, the self-knowledge and transformation are dependent on personal confrontations escaping the limits of an impersonal world.

Many weaknesses can arise from self-invention and non-conformity.

The self-knowledge and transformation are depended upon personal confrontations escaped the limits of an impersonal world.  The personal world is one that is usually independent from the way individuals seek and understand more subtle movements in the world, the traits that 


These values embedded in the Romantic world drives human thoughts, helping us realize a certain invention

\fi

Self-invention and nonconformity, when pursued in excess, have the potential to incur delusion and disillusionment.  A classic example would be the story of Miguel Cervantes's Don Quixote, who misrepresents certain realities of the world as obstacles to overcome.  For example, Don Quixote mistakes windmills for ferocious giants and designates Aldonza Lorenzo, a neighboring farm girl, as a princess.  

Don Quixote exhibits self-invention by the way he defines his challenges and by the way he proceeds to overcome these challenges.  Don Quixote is indeed striving for faith, hope, and love in his own world, a striving that is consistent with self-invention in the Christian-romantic sense.  Moreover, because Don Quixote contends against these challenges, Don Quixote exhibits nonconformity by not giving in to the pressures of these challenges, to ultimately not accept defeat.

While Don Quixote exhibits nonconformity and self-invention in this sense, his nonconformity and self-invention does not feel quite right as how we interpreted nonconformity and self-invention in the Christian-romantic sense.  It is clear to the reader that Don Quixote lives under a certain delusion where he misrepresents his relationships with the world, a misrepresentation that makes him look less like a hero on a journey but more as an insane individual creating chaos in the world.

When reading Don Quixote, we come to understand that while Don Quixote is trying to create faith, hope, and love in his own world, he is unable to instill faith, hope, and love in a broader sense, one where the worlds of everybody around him are all instilled with greater faith, hope, and love.  We come to understand that while maintaining and striving individual ideals is commendatory, these ideals are worth striving up to a fault.  When these ideals are inconsistent with moving the faith, hope, and love of the world at large, perhaps these ideals were more self-delusional.

In addition to the weakness of self-delusion in nonconformity and self-invention, a major problem is also the possibility that the ideals and goals harbored by the individual actually produces no true triumph of happiness for the individual, leaving the individual a ``hopeless" romantic.  

One of these mishaps and illusions lies in the temptation to recreate the past amidst the struggles of the present.  We can view Don Quixote's journeys as the projection of inner desires to realize past glories and stories akin to that of King Arthur and the knights of the round table.   Don Quixote, rather than assessing the true nature of his situation and how the times have changed, lives with his sights set on the past.  By doing so, Don Quixote is unable to truly realize a unique vision of faith, hope, and love that is isolated from the constraints of past events.  In this way, a statement is made that in order for faith, hope, and love to possess its true meaning in the Christian-romantic sense for the individual, it must take on a timeless quality, transcending past precedence and also transcending future obstacles.


\section{Nonconformity and Self-Invention in the Career}

Just as how modernism is a transformation of Christian-romantic ideas, so too is the modern career a transformation of the hero's journey.  While technology, politics, and culture have significantly changed from the times of Christian-romanticism, the fundamental challenge of the individual trying to realize ideals of hope, love, and faith amidst impersonal societal obstacles still remains the same.  

A fulfilling career has the capability of establishing hope, love, and faith amidst impersonal societal obstacles.  The career is characterized by a search for a justified task where the individual is able to elevate his craft and devote his relationship to a certain form of work, asserting his self-invention by the quality of work he produces.  This career is often at odds with the conventions of modern society, where the individual's aspirations do not conform to the guidelines of work in present-day society.  As a result, society will initially reject the individual's work, resulting in obstacles that prevent the individual from truly realizing career aspirations.

At the same time, if the individual becomes too self-absorbed in his career, he is prone to experience the weaknesses of self-invention and nonconformity, becoming a slave to his work, driven by obsessive tendencies.  These obsessive tendencies are no different than delusion, for the individual has constructed a world that has deviated so far from the realities of hope, love, and faith of the broader world such that the individual's world ceases to possess meaning any more.  Such a career prospect is not desirable.

Ultimately, the full narrative arc of Christian-romanticism manifests itself by how the individual maintains his convictions in his career aspirations, perfecting his craft without overextending his pursuit of craft into that of obsession.  It is with these ideas of balancing the strengths and weaknesses of nonconformity and self-invention that allows an individual to live meaningfully through a fulfilling career and to embark on a
hero's journey in the modern world.

\ifdraft

 deeply in work in order to truly transform the capabilities the context-transforming capabilities

pursued with a sense of urgency that allows an individual to emerge 

as a description of a recoverable past rather than the dreamlike expression of 
constraint and insufficincy in the present.  The creation of a recoverable past is 
what has plagued man individiuals in their working career, for examples 
individuals dwell upon the ``golden times" of college all too bitterly that they 
fail to appreciate the beauties and opportunities that lie at the present 
moment. 

will make an impact on society.  While impact 

At the same s

 hero's journey in the modern world.

The distinction between job and career is made.  


 how the narrative arc is created to make the individual

schanging past prejudices of social pressures 

sdoubt where each candidate activity derives an apparent worthiness from the groundless prejudices of particular culture, nothing more than the arbitrary symbol of a need for justified action.  This search search for a  task that would be worthy of him, worthy above all of his context-tran- scending identity and his context-transforming capabilities, the search for a perfect ``job" rather than the search for a more fulfilling ``career," where the full narrative arc of the Romanticism presently manifests itself.

Weaknesses of self-invention is that in love as in work, the protagonist of late romance seeks an escape from the dangers of solipsism and self-obsession. 
love becomes a struggle for exit from the self and for sustenance from others amid the deceptions and oppressions of society.




The modern career is the most


s with Homer's \textit{Odyssey} and the medieval \textit{King Arthur}.

 intrepret arcadian myth 
 The literary-psychological counterpart into the politics of the natural 
context is the effort to regain the moment of visionary


This could best be seen with how late romance presents an important idea in Wester

 A standstill arises where individuals are unable to accommodate



Such are the perils of nonconformity and self-invention



Late romance presents an important idea in Western prose fiction and poetry from mid-seventeenth century to the twentieth century in that it is early romance with an ambivalence with anxiety over whether such goals are worth striving for are possible of materializing in the world.

Many mishaps and illusions present itself under the values of the late Romantic period.   




A major problem with adhering to self-invention and nonconformity would be the creation of a hopeless romantic.  The hopeless romantic proceeds about with an 

A major weakness characteristic of self-invention and nonconformity could be the possibility of creating a hopeless romantic.  The hopeless romantic is one where the individual goes along with an  


The hero is able to make a home out of the renewed earth as true identities are revealed and marriages are celebrated, and new orders of social life is generally established.  Rather than becoming a hopeless romantic, the individual has to go along with an unembarrassed confidence in conventional morality and hierarchies of value.  The downsides may very well be the individual does end up becoming a hopeless romantic.  Late romance presents an important idea in Western prose fiction and poetry from mid-seventeenth century to the twentieth century in that it is early romance with an ambivalence with anxiety over whether such goals are worth striving or are possible of materializing in the world.

Late romance possesses specific literary origins such


Mopernism and modernist are used here to designate a specific movement of opinion and feeling rather than a vague sense of contemporaneity

They easily mis- take the deficiencies of a particular social order for the inherent limitations of society. And they repeatedly find that the search for personal experimentation and self-fulfillment ends in disappoint- ment when it is not tied into a wider social solidarity or a larger historical project that can rescue the individual from his obsessional and futile self-concern.

no institutional order and no imaginative vision of the varieties of possible and desirable human association can fully exhaust the types of practical or passionate human connection

The result of these tendencies has often been to turn the modern doctrine of contex- tuality into the belief that the individual can expect no real progress from the revision of his contexts. He can assert his in- dependence only by a perpetual war against the fact of contextu- ality, a war that he cannot hope to win but that he must continue to wage.

WEAKNESS exaggerates certain elements of the tradi- tion while suppressing others

But it ends as an inability to see the other individual as a person with his own resources of secrecy and striving. It therefore amounts to a denial of that imagination of otherness upon which the entire life of society draws, this leading to depictions like Don Quixote.

 acknowledge that Christianity and other 

A legitimate system of hierarchies and stations leads the the disruption of social roles, divisions, and hierarchies, one where a social order is the one making itself more completely accessible to preventing roles, divisions, and hierarchies.  


to be true and good in the individual's own justification of how to proceed with life.  Moreover, a person's deepest identity is not defined by membership in social ranks and divisions.  These


The adoption of certain values accepted to be true and good can be largely interpreted as the conforming to 

Nonconformity is a representation of 

 illustrates an inherent bias of the individual to have already conformed to certain standards of society.  Moreover, the 

Modernism, as Unger describes, is built upon the historical appreciation of transformations characterizing Christian tradition.  Ultimately, it is this modernism that defines the modern world where nonconformity and self-invention.  

The modern world is a purification of the Christian-romantic idea of the self.  The modernist picture of personality 


In the modern world, a career is best defined to be, ``a calling requiring specialized knowledge and often long and intensive academic preparation"\footnote{defined in Merriam Webster Dictionary}.  There is often discussion between a 

We are operating with the belief in a natural context for social life is likely to go together with the convictions that there exists a fixed, ideal balance between the claims of engagement and of solitude and that a specific set of social arrangements- almost invariably an idealized version of present society- realizes this balance

This distinction between career and job is important because it better contextualizes the focus and benefits of competing goals between nonconformity and self-invention in the modern world, specifically the concern of individuals not being able to discuss the thoughts pertaining to human understanding and human capital.  
\fi
\ifdraft

A career implies speech input.  Inverting the speech patterns of input would necessitate different structures of human reasoning and human ability.  The structure would crank out correct output.  The is that occurs after tree is here to ask the question and we implement a mental algorithm.  The argument is that disambiguating requires something more fundamental.  Rare child inputs

In the Romantic world, the individual is the victim of misfortune.  Paganized versions of romance demonstrates an escape won through patience and guile, where good will and grace presents a proliferation of opportunity that extend beyond human forms.

Moreover, there is also a difference and distinction between human beings and infallible creations.  These questions make us more questioning and doubtful of the original intent of human understanding.  These

This specialization of knowledge involving long and intensive academic preparation is often a precursor.
Non-conformity could best be described as a 

\section{The Career}
Understanding the strengths and weaknesses of self-invention and nonconformity suggest elements of a greater career, one where individuals are capable of looking at the broader picture.


The liberal arts college and passion as Professor Unger
writes, is a seemingly incoherent piece of literary
analysis.  What is passion or novelty if we can't for
somehow turn it into actual machines capable of thought?  We
would otherwise be devoid of expression, losing sense of
what it means to write broadly about our own experiences.
Such difficulties always have a certain implication in how
I write code, or how I interchange different things
together, but together, they always piece together
something that is greater than myself.  For that reason, I
find myself completely interchanged into the analysis and
understanding of life.

What more can I really say about this situation that is
that my muscle memory is rather short and that I have
rather limited ability to write or expound certain ideas,
the text is limited and I have found myself at a loss in
saying what truly is meaningful.  Having learned from my
mistakes, I can now proceed with the rest of my journey in
greater fashion and greater understanding.

The world we live in, thus presents us with a renewed
awakening for the many other qualities of life.  We are not
at all burdened by status.

There is some difficulty in saving files in a way that works for many people.  I for one am confused broadly about all these notions behind what we do and why do we do it, these are great options that have fueled me to greater understanding.

I have also just wondered briefly what it means to form your own identity.  To live out a personalized vision in a sea of distress.  What does it mean for somebody to become wholesome?  To live according to virtue?  These are all larger questions of the world, to seek out for once meaning and the lack thereof.

It is moments like these that make me more appreciative of what other services have provided me but I too am at a loss sometimes to understand what is great and true.

\fi
\end{document}
