\documentclass[12pt,letterpaper]{article}
\usepackage[utf8]{inputenc}
\usepackage[english]{babel}
\usepackage{amsmath}
\usepackage{amsfonts}
\usepackage{amssymb}
\usepackage{graphicx}
\usepackage[margin=1in]{geometry}
\usepackage{setspace}
\doublespacing
\author{Justin Zhu \\ Advised by Dr. Alexandre Victorovich Gontchar \\ Written for \textit{Conduct of Life}, taught by Professor Unger and Professor Puett}
\title{The Academy and its Aftermath: Internet Meme Culture and the Subsummation of Liberal Arts Education}
\date{}
\newif\ifdraft
\pagestyle{empty}
\begin{document}
\clearpage\maketitle
\thispagestyle{empty}
\pagebreak
\tableofcontents
\pagebreak
\pagestyle{plain}
\setcounter{page}{1}


\section*{Preface}

In Harvard's Langdell North Hall that iconic portrait of Rosa Parks hangs.  That prophetic gaze is surrounded by that of other notable Harvard alumnus, their portraits impressible, stimulating the better natures of ourselves.  

This is the type of place where great thinkers are created, it seems.

Every Wednesday from one to three, undergraduates from the College, law students, and Harvard affiliates file in for \textit{Ethical Reasoning 20: Conduct of Life in Western and Eastern Philosophy}.  The main attraction is Professor Unger and Professor Puett, two giants in their fields -- their work and discourse complement so well together, I can think of few partnerships that proved so edifying and apolitical.  I was in luck this particular semester because I was taking classes taught by dynamic duos -- there was operating systems by Professor Kohler and Professor Mickens, and inference by Professor Blitzstein and Professor Murphy.  Two professors are indeed better than one, just as two parents to one.  Three would be even more ideal, as Confucius would say.

Unger's fiery bravado and Puett's nuanced polysyndetons bring out the full contrast of human eloquence, played out in philosophical layers.  Considering the class is titled ``Conduct of Life According to Eastern and Western Philosophy," it's tempting to characterize Unger as the ``West" and Puett as the ``East."  Instead, as the conversations unravel onto themselves, one begins to see that the two thinkers share congruent notes on many of the most pressing problems of the human condition, vis-a-vis the ``Western" and the ``Eastern" approach.  

Certain examined truths are timeless, it seems.

As I made my way towards the very front of the room, I walk past the Facebooks, the Gmails, and the Twitters that occupy the screens of those who have come in this warm Wednesday afternoon.  As class begins, no context switch occurs -- the cold screens continue to flash the Facebooks, the Gmails, and the Twitters, and even Unger's earnest assertions (``Courage is enabling virtue without which all virtues are left sterile!") seems to lose its effect -- the student to my right furtively tagging her friend on a meme.   As the professors open up the room to questions, silence ensues.  As if Schopenhauer was a vocabulary word!  

\ifdraft
And I too was one of them, and there I looked, just couldn't help but noticing a certain look in Puett's face that can best be described as pity.
\fi

``God is dead," proclaimed the great Nietzsche.  Is so too the liberal arts education?

\section{The Liberal Arts Academy}



There is also an interesting overtone to which the liberal arts education can be perceived as deceiving.  I dor onw have often realized this to be the case, that our thoughts are foten enjoined with the prospect that life, no matter how short and unrefined, still  have other fac

as I defined in my first paper, ``Self-Reliance in the Digital Age."  The class, as with all other classes at Harvard, adopts a very lenient policy

Such a scene expresses the liberal arts culture as I experience it now, as an undergraduate who once had some aspirations to leave my mark to humanity in the humanities.   



They click on thumb icons, heart icons, tagging their friends on Facebook.  A conditioning that is inevitable, but also as a way to express ourselves and assert our existence.  The power of the Internet is its connection, and it's this connection that fills me with great optimism for the world ahead.  The marked departure to the very ideas expressed by all of us.  These are the essays that define our generation.

Looking at the students 

The sight of an entire auditorium nowadays sprawled with new is not an uncommon sight.  The content of these screens is furthermore interesting.  We can examine the psychological narratives of our own world departing from the limits of the human imagination, that is, wes

 many rows and rows of students whose eyes dart forth on their laptops sprawled with Facebook, Gmail, and Twitter.  This is the world we live in, a time where the distinction between work and play blur, and the precious seconds of our lives are spent.  Notions of the self are changing, and this is the paradigm that has made us increasingly less hesitant that we know what we are doing than to doubt our inability to do it successfully.

when Harvard is known as the ``Stanford of the East," which is not an affront to Harvard or to Stanford, but to the explicit infallibility of the liberal arts culture and the human tradition giving way to engineered systems and machine learning.  We are not as ``sophisticated" as we make it to be seem.  What's to stop a machine learning algorithm to do all

As class starts, no context switch occurs -- everybody proceeds to refresh their screens, checking the NewsFeed on their Facebook profile.  While they might.  These images are some of the 

the content of Puett and Unger are rapidly profound, the effect is not particularly so.  

Few were taking notes on their MacBook Pros, and there was not a single screen where I did not see Facebook, Gmail, or Twitter occupying a third of the screen.  I defined in my first paper, ``Self-Reliance in the Digital Age," that the underpinnings.

\section{The Academy}

I remember distinctively as a, eager to drink deep, lest I taste not that Pierian spring.  I was shocked when Professor Unger in the very first lecture of the class immediately.  If philosophy, the mother of all humanities had descended into death, then how can I truly have witnessed the own subsummation of my own culture and my own anticipation for the way things go?

 What I witness everyday is the Internet Meme Culture, and the subsummation of the liberal arts curriculum.  Just as Modernism defined and, the current post-post-modernist landscape is captured best by machine learning and memes, a cyclic loop of content and optimization, leaving little room in the liberal arts.  A whole generation staring endlessly through 





\ifdraft
My class schedule this entire semester had taken on an interesting flavor -- it was going to be taught.  The benefits of, it's like growing up with two parents who aren't divorced.


I can name of few dynamic duels that I know off -- Joe Blitzstein and Susan Murphy in the Statistics Department, James Mickens and Eddie Kohler in the CS department, whose operating systems class starts at three that very same day.
  \fi


The class is ``Conduct of Eastern Philosophy."  Ooh, I should try it as if it were some type of East Asian delicacy.

This place in the year of 2019 tells of a distant story.  Shelley's Ozymandias poem.  Look upon ye mighty and despair!  And this has been the general sentiment.

Yet, a friend of mine Gabriele.  Want, He's a good friend, and I will certainly chanting Hebrew on Saturday mornings when I think about my time back here at Harvard.

A certain type of gloom hangs in the air.  Philosophy, the mother of all humanities sounds nothing but a sound of platitudes.  

Unger bellows, ``Courage is enabling virtue without which all virtues are left sterile!"  Besides me, a student refreshes Gmail.


 Such a scene would make anybody exclaim, ``Wow,  This truly is Harvard."

For example, a portrait of Rosa Parks hangs in the amphitheater surrounded 

\section{The Liberal Arts Tradition}
To be fair, Harvard was never particularly successful in getting its general education system up and running.  In 1916.

Drew Faust

The uniqueness of Harvard's liberal arts tradition has always been to cultivate ``gentlemen-leader of the world."  In the current post-modernist phase, we would be quick to point out that now know under, a very gendered term and also very colonialist, and colored by capitalism overtones.  But such a critical attitude breeds cynicism.

\section{The Internet Meme Culture}
In 1976, Richard Dawkins published ``The Selfish Gene," explaining how cultural information spreads.  In ``The Selfish Gene," Dawkins coined the word meme as we know it today.

Faust left for Goldman Sachs.  The irony and literary allusions surely did not leave us.  The meme culture.  A cultural tradition has always been dependent on shared appreciation.

.   It's the six.  Unger bellows into the air.  It's somewhat unsettling that the very auditorium, a certain despondence hangs over the classroom.  This is a class with law students.  I peer over, glancing.  Beyond me, another student copies and pastes his resume onto his LinkedIn profile.  All seem impervious to the lectures of 

It had been a Harvard I had seeking.  Going about.

Harvard, the vestige of high American culture, high world culture seems to run hollow these days.  God is dead, proclaimed Nietzsche, and perhaps too is the liberal arts curriculum.

\section{A Time of Recruiting}
It doesn't come as a surprise.  The liberal arts.  Part of the fascination with corporations is that they have logos, those picturesque pieces of, or that certain ``hue of blue" (referring to Goldman Sachs or McKinsey).  With no measure of true excellence endowed in the academic sphere, ambitious students search for excellence in other areas, which certainly falls in the.  

These logos of corporations have themselves been part of the meme culture.  Google's very own ``Google for Doodle."  Forget about surrealism or the arts.

One is left to wonder what is the state of the liberal arts curriculum in this day and age, whether any of it exists at all.  The influx of computer science [insert citation] moreover, is illustrative of a larger issue that the liberal.

\section{Asian Culture}
Another point I would like to expound on is the Asian culture, which has always been copy what works and then make it more efficient.  This can be seen in the.  Western paradigms of artistry and individual ownership of work are not pervasive themes in Asian culture, which I should discuss mostly as culture in the Chinese, post-Mao tradition.  Mao's influence on China has one that led to a ``survivalist" instinct in the masses.  Those who survived.  My grandparents have been very reticent about this time in their lives, and from what I gathered, it has been no different from Cormac McCarthy's \textit{The Road}.  When I return to China, I have often felt.  My friends often asked what China was like.  

Imagine capitalism without Christianity, I said.

I bring up Asian culture because it largely reflects my own.  I have always identified as Russian, since my mom was raised in the very northern regions of outskirts.  While it is the last names of our fathers that we carry in our names, it is the soul of our mothers that we carry in our lives.


\section{Shakespeare and the Humanities}
Harold Bloom commented on 
I originally planned to concentrate in English because it represented more of my own personal 

We are left to ponder what then consistutes the formation of the self.  Individuals like Hamlet.  The problem with Hamlet is that he doesn't say what he means and doesn't mean what he says.  And I too, like Hamlet, have found myself at odds with myself, purporting that I embody a certain sophisticated individualism, while at the same time recruiting shamelessly for investment banking firms, hedge funds, and big tech companies like many.


\section{The Post-War Asian Culture}
Puett describes.  

But this is no counsel of despair. In 1773, Samuel Johnson visited the University of St. Andrews on his journey to the Western Isles of Scotland. St. Andrews is an ancient institution, one of the 25 or so oldest universities in the world, and yet 350 years in, it had evidently fallen on hard times. Fewer than 100 students remained, and one of its old colleges had been dissolved. "To see it pining in decay and struggling for life," Johnson noted, "fills the mind with mournful images and ineffectual wishes." He was under no illusion as to where the blame lay: "It is surely not without just reproach, that a nation, of which the commerce is hourly extending, and the wealth encreasing ... while its merchants or its nobles are raising palaces, suffers its universities to moulder into dust."

The humanities and the university do need defenders, and the way to defend the humanities is to practice them. Vast expanses of humanistic inquiry are still in need of scholars and scholarship. Whole fields remain untilled. We do not need to spend our time justifying our existence. All we need to do is put our hand to the plow. Scholarship has built institutions before and will do so again. Universities have declined and come to flourish once more. The humanities, which predate the university and may well survive it, will endure — even if there is no case to defend them.



\section{College}
Universities have become a big social club where.  Colleges are not about examining timeless truths.  Universities are not about building skillsets, at least in the humanities and social sciences.  Instead, universities are prioritizing credentials and big social clubs.  There is a vast impact on American society, the intellegensia, the foolish attitude of the white, working-class who did not go to college.  People don't take out their IQ scores.

The way people think about college is "He want to Harvard.  He must have been very smart."  This right-wing figure were over a conflict over a film.  This person said to Jeremy, you didn't even go to college.  People who go to college feel like they are more privileged.  The person who went to Yale is smarter than Junior college.  It is not universally true, it does not translate over to a level of success.  Innate ability does not translate to ability.

People can brag about where they went to school.  Most of us have institutions.  Church and work.  Their social fabric is created by them working with others.  School or church.  As churches declined and as schools become less of a binding committment, then social fabrics disappear.  Colleges have become a new social fabric.  



\section{The Academy}
The liberal arts curriculum is one often represented by multiple stages of human underpinnings.  The motivation, life as we know it is propelled into something deeper, more elemental.  Such are the purpose.  This paper is a dedication to a journal, manifesto of personal philosophies compiled and influenced by Nietzsche, Schopenhauer and other notable thinkers.  It is a tribute to the idea of human striving, of the mere ``essentials" of life, of tasting the bare essentials of life.

A major question has always been the purpose of the academy, of higher-education, the very liberal arts curriculum that an individual undertakes during the formidable years of college.  This education is formidable because it marks the supposedly complete maturation of the self, one where the individual will have explored deeper ideas of his own identity and personality.  The thinking goes that once a young individual is able to fully identify his/her own identity and personality, the individual will be able to identify .  The job market comes and goes, with new jobs created and destroyed all the time.

\section{Sun Tzu: The Art of War}
The Art of War is pretty important in that the individual is able to internalize something more 

\section{Youth and Self}

The youth is what is most impressive, characterizing the most important attributes of human surroundings.  Sometimes I woder who I am, the effusive underpinnings, lost in thought.

\section{Relationships}
The Confucius tradition necessitates a certain grounding of truth, where individuals are.  Confucianism is a broader set of ideas where individuals are compelled to look upon

\section{Individualism and Self-Reliance}
This is the individual and self-reliant perspective, where individuals are able to internalize and describe something more fundamental.  Individuals are able to internalize and describe the world, seeing a more versatile literary style.  People are more contrived to explore other pathways to human understanding.  My own personal manifesto.

\section{Work and Family}

\section{Romance}

\section{Principles}
Courage is the enabling virtue by which all other virtues fall apart. 

\section{Schopenhauer}
Jettisons everything, creating lead causality,  if you look  at the system of arts.  Causality is how we cognitively structure.  How we engage with the world artisitcally.  Defeat and offer another paradigm that is outside causality.  It's a fine thought.  Music is the highest form because it doesn't have the connection to the materiality of the world.  He doesn't represent everything.  This is similar to what conecpts are.  It's that some meaning are always connected in our mind, a string of meanings that lead us to create certain judgments.  Profoundly disconnected from the mateiral world of stuff.  A certain ontology of the world is offered ti the third world.  You can think aof everything you experience as having a liftime of its own.  We define life ontp a life of the function.  Things have their duration, such as a table tahat comes to life as a table.  Everything like the chairs.  This will is an ocean of energy that is self-contained, branching into multiple things.  In the worl dthat we inhabit we see many things have the duration that they do.  Everything is regent as we know it in the world.  Every one of us is a microcosm, everything one is a notion of ocur kind.  And in that sense, if we think about the self, we too exist, and how we live or do like these processes

What is the idea> The idea is that we do not to  It's only the ethical register where we wake up face to face with how the world fundamentally is.  Here we can do X but we are not going to do X, and so we inhibit that impulse.

I don't think there is a thing like Kant's second critique.  Kant is cloeset to the ancient thought, not every one of ius is different.  There's a way the world is primordially structured.  Plato and Aritotle is to see that eternal structure of the world, and get it from God's point of view.  Once you have that access, then you will be able to just do the right thing.  No Steve JObs, or Napolean.  They will just disregard the idea.  The narrative sself having the ontological weight.  It's a strange eclectic with Aristotle and Kant.

Nothing happens with this chari to sit on, in multiple spaces, there is no chair.  It was always there.  Everything we see in the world is an illusion, the smotley world of entities, things we could not be clinging to.  Look, you have to understand this hierarchy of being.  Here is where we offer a Buddhist take, where the best thing to do is to not do anything.

Kant's critique, Aristotle and Plato.  Philosopher is a mechanic of the soul, drawing out the talents of people, so that people would open up.  This is a good Buddhist thought, not as good as somebody with a high level of.  Systematic thinkers with eclecticism like Marx, Hegel.  Some aspects 

Schopenhauer is writing from a state of unhappiness, in contrast tot he Buddha, who is cominng from happiness.  Contrasting those dimensions in a very present way.  This juxtaposition between the two.  

\section{Liberal Arts}
The liberal arts education isn't dead so long as human beings are not dead.  The academy. 

Yet, what is so imperative to the structural arguments that characterize our abilities to speak deeply about our experiences, it is the construction of these systems where computational metrics proliferate and we run the risk of failing to account for the world.

Like taxes, the liberal arts education requires us to give up a bit of our personal comfort to live a life that is less wasteful.  Those timeless truths still apply, but are not so readily apparent or readily incentivized in our modern day and culture.


\end{document}
