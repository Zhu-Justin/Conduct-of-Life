\PassOptionsToPackage{unicode=true}{hyperref} % options for packages loaded elsewhere
\PassOptionsToPackage{hyphens}{url}
%
\documentclass[]{article}
\usepackage{lmodern}
\usepackage{amssymb,amsmath}
\usepackage{ifxetex,ifluatex}
\usepackage{fixltx2e} % provides \textsubscript
\ifnum 0\ifxetex 1\fi\ifluatex 1\fi=0 % if pdftex
  \usepackage[T1]{fontenc}
  \usepackage[utf8]{inputenc}
  \usepackage{textcomp} % provides euro and other symbols
\else % if luatex or xelatex
  \usepackage{unicode-math}
  \defaultfontfeatures{Ligatures=TeX,Scale=MatchLowercase}
\fi
% use upquote if available, for straight quotes in verbatim environments
\IfFileExists{upquote.sty}{\usepackage{upquote}}{}
% use microtype if available
\IfFileExists{microtype.sty}{%
\usepackage[]{microtype}
\UseMicrotypeSet[protrusion]{basicmath} % disable protrusion for tt fonts
}{}
\IfFileExists{parskip.sty}{%
\usepackage{parskip}
}{% else
\setlength{\parindent}{0pt}
\setlength{\parskip}{6pt plus 2pt minus 1pt}
}
\usepackage{hyperref}
\hypersetup{
            pdfborder={0 0 0},
            breaklinks=true}
\urlstyle{same}  % don't use monospace font for urls
\setlength{\emergencystretch}{3em}  % prevent overfull lines
\providecommand{\tightlist}{%
  \setlength{\itemsep}{0pt}\setlength{\parskip}{0pt}}
\setcounter{secnumdepth}{0}
% Redefines (sub)paragraphs to behave more like sections
\ifx\paragraph\undefined\else
\let\oldparagraph\paragraph
\renewcommand{\paragraph}[1]{\oldparagraph{#1}\mbox{}}
\fi
\ifx\subparagraph\undefined\else
\let\oldsubparagraph\subparagraph
\renewcommand{\subparagraph}[1]{\oldsubparagraph{#1}\mbox{}}
\fi

% set default figure placement to htbp
\makeatletter
\def\fps@figure{htbp}
\makeatother


\date{}

\begin{document}

\hypertarget{emerson-and-the-conduct-of-life}{%
\section{Emerson and the Conduct of
Life}\label{emerson-and-the-conduct-of-life}}

In `Self-Reliance', Emerson discusses the fundamental importance of
asserting the individual self, comprising of several steps such as
speaking your latent conviction, and the like.  Emerson is unable to discuss deeper, truer implications for his actions and for his words.  These implications arise out of steadfast understanding, life in the horizons as we are comfortable with.

\hypertarget{emersons-weaknesses}{%
\subsection{Emerson's weaknesses}\label{emersons-weaknesses}}

\hypertarget{addressing-those-weaknesses}{%
\subsection{Addressing those
weaknesses}\label{addressing-those-weaknesses}}

\hypertarget{conduct-of-life}{%
\subsection{Conduct of Life}\label{conduct-of-life}}

Emerson writes that, ``But do your work, and I shall know you. Do your
work, and you shall reinforce yourself.''\\
In an ideal world, everybody is able to find meaning in the work that
they do, deriving meaning from this work. We often think about
professors, CEOs, and these sorts of individuals who are very much
privileged people in the top echelon of society, but rarely do we
consider the other half, that is, the people who are not at all
accustomed to the unknown, that we are working towards a life that we may be somewhat unsure of.  Mathematics, computer science, and everything else presented into its full frontal form.

However, only a certain number of people have the ability to work
towards making this work a reality, and that itself is the
infeasibility, the impossibility of Emerson's goals.

How can we best reconcile what Emerson is speaking? How Emerson is
saying it? It is very much our own personal biases, judgments shrouding
our ability to speak upon such matters, and that itself is often
unsatisfactory. To understand these large philisophical studies and

\end{document}
