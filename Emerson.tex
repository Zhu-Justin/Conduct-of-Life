\documentclass[12pt,letterpaper]{article}
\usepackage[utf8]{inputenc}
\usepackage[english]{babel}
\usepackage{amsmath}
\usepackage{amsfonts}
\usepackage{amssymb}
\usepackage{graphicx}
\usepackage[left=1in,right=1in,top=1in,bottom=1in]{geometry}
\author{Justin Zhu \\ Advised by Sasha}
\title{Self-Reliance in the Digital Age}
\date{Paper for \textbf{Conduct of Life}, as taught by Professor Unger and Professor Puett}
\newif\ifdraft

\begin{document}
\maketitle

\section{Introduction}
In this paper, I will reconcile Emerson's vision of
self-reliance with certain challenges of conducting life in the digital age.
This digital age has witnessed fundamental changes
in human behaviour and human interaction via the rise of
social media and the projection of a digital self, where
this digital self is divorced from raw reality.  This
digital self is often based on idealization rather than
reality.  The rise
of social media platforms like Facebook has taken hold of
many individual's time.  Recent studies have suggested that
human beings are spending most of their time on the
computer.

From an "anti-self-reliance" perspective, the conduct of individuals are a
social media landscape taking hold of individuals' time.

These realistic goals of individuals in the
digital age primarily consists of a social media landscape
where individuals are capitulate by partaking in social
media.  


\section{Writing Efficient Algorithms}
I think basic software is relatively easy for me to achieve.
This is coming from somebody with minimal CS experience.

\section{Axioms}
First, I will establish some axioms.  The term "digital
age" is an umbrella term that refers to a time period
where interaction with screens is made possible.  That is,
human behaviour has been chiefly oriented around consensual
interaction with other human beings.

\section{Falsehood of Self-Identity}

\section{Social Media and the Woods}

\section{Deep Work and Emerson Identity}

where individuals find themselves capitulating to social media
and constructing a digital self in the modern age.  I will
examine particular dichotomies, which can be largely
interpreted as failings in Emerson's argument in
anticipating exceptions seen in our time that he might have
failed to account for during his time.

The great society that Emerson writes is one built on past
ideals of correctness and sincerity.  The life outside of
the confines and many other things leave us often utterly
convinced that there is a greater solution.

Rather than accepting that individual contributions are
often at odds with the rest of the society, the opinions
fostered by most individuals are creating something that is
more egalitarian.

The society that Emerson writes about in his Self-Reliance work is one ingrained in fundamental principles of nature. How come?  What is the natural world that Emerson writes about?

This is moreover the setup I've enjoyed.  I think it has
presented itself with many fantastic benefits.

Every Mac Desktop and Windows PC comes with a default desktop background of a natural landscape.  The world of the 21st century is completely digital, and human interaction may be part of a bygone era.  Is this a falsehood that governs our day-to-day activities or is this something more fundamental?

The idea that Emerson has contrived is one where true work is independent with the other notions of identity. 

In 'Self-Reliance', Emerson discusses the fundamental
importance of asserting the individual self, comprising of
several steps such as speaking your latent conviction, and
the like.  What is this conviction?

Moreover, individuals are rarely able to reconcile the good
and the bad.  This has been the many wonderful qualities
that have governed modern human existence.  There is finally
a question of pertinence that permeates much of human day
existence.  Why should Emerson or Nietzsche be here to
criticize or remark on these human virtues?

Evince is a great document viewer, and it's something I've
been using for all my classes.  I love to code and I love to
type out things that make me a more efficient writer.  This
is the benefit of computer science and English narration
captured in its most fundamental form.

I have often stopped to remark that these things, while good
and true often are not enough to satiate human desires,  it
would be interesting to comment on the feasibility of these
tasks at the end of the day and what qualities are the most
conducive for making such things like this possible.

The Emerson way would be to completely abandon social media
altogether.

\section{Emerson's weaknesses and addressing these weaknesses}

Is nature if ideas inspire them?  Are those ideas legitimate?  What makes us say that these ideas are part of what we want to do or say?


\section{Privileges}
Broadly speaking, Emerson's ideas also attest to a
fundamental aspect of human nature, which is our own
inconceivable understanding of humanity.

Emerson writes that, "But do your work, and I shall know you. Do your work, and you shall reinforce yourself."  But what is this work that Emerson talks about?
  
In an ideal world, everybody is able to find meaning in the
work that they do, deriving meaning from this work.
We often think about professors, CEOs, and these sorts of
individuals who are very much privileged people in the top
echelon of society, but rarely do we consider the other
half, that is, the people who are not at all positioned to think about their own lives.

However, only a certain number of people have the ability to
work towards making this work a reality, and that itself is
the infeasibility, the impossibility of Emerson's goals.

How can we best reconcile what Emerson is speaking?  How
Emerson is saying it?  It is very much our own personal
biases, judgments shrouding our ability to speak upon such
matters, and that itself is often unsatisfactory.  To
understand these large philisophical studies and 

\section{Digital Self-Reliance}

The notion of identity has become one that is intertwined with other aspects of other identity.  This identity is one that is not solely divorced from other notions of what is right or wrong.  The virtues that Emerson writes about is largely independent with that of other virtues.  Chiefly speaking, our ideas on the world and how we are to mold it into our very own creation can be seen to be tangentially related to what we describe as the optimal virtue, a world where our efforts are largely independent with that of other students.

\section{Meaningful Work in the Realm of Computational Thinking}
The computers where we think about meaningful work is often ill-contrived.  Not many people are well-maintained. 

In ``Deep Work," Newport writes about how his work is entrenched in basic principles.


\section{Nietzsche}

Romantic quest has the genius vs. them the crowd, the others.

Celestial body.  Stendhal's Red and Black is Christian because it involves the fall.  You read books, that's why you find yourself more compatible.  Then you fall in love.  Several things must happen.  There is definitely a falling away, falling from the natural habits according to your class standing, education, et cetera.  Stories that are Biblical to their core.

How that rebellion takes place, the possibility of imminent critique.  It's not until late 60's.  There is another version of this, an idealistic person comes with a simple realization in the end.  Understand the way things are run.

\section{Bound}

Spirited, Rational, Appetitive

Harmonic Development

Privatized Sublime


For Plato, his Republic is all the judgment plains.  If he can approach every situation he is in.  


\end{document}
