\documentclass[12pt,letterpaper]{article}
\usepackage[utf8]{inputenc}
\usepackage[english]{babel}
\usepackage{amsmath}
\usepackage{amsfonts}
\usepackage{amssymb}
\usepackage{graphicx}
\usepackage[margin=1in]{geometry}
\usepackage{setspace}
\doublespacing
\author{Justin Zhu \\ Advised by Dr. Alexandre Victorovich Gontchar \\ Written for \textit{Conduct of Life}, taught by Professor Unger and Professor Puett}
\title{SELF-RELIANCE IN THE DIGITAL AGE}
\date{MARCH 13, 2018}

\newif\ifdraft

\begin{document}
\maketitle
\pagebreak
\tableofcontents
\pagebreak
\section{Thesis Hi}
In this paper, I will reconcile Emerson's vision of
self-reliance with certain challenges of conducting life in the digital age.
This digital age has witnessed fundamental changes
in human behavior and human interaction via the rise of
digital worlds and the projection of a digital self where
this digital self is divorced, and often deceiving, of the real, true self.    The rise
of social media platforms like Facebook and messaging services like Gmail and Slack have fundamentally altered human identities, ultimately distancing human beings from the self-reliance that Emerson advocated for.  Emerson's ideas are not a solution for conduct of life in the digital age -- in fact, I will identify in subsequent sections that certain ideas of Emerson's self-reliance are fundamentally incompatible with such realities of everyday life.  However, being said, certain Emersonian self-reliance are still relevant and, moreover, prescriptive for living better lives in this digital age by nature of encouraging individuals to become independent thinkers.  Thus, a reconciliation of Emersonian ideas and of communication in the digital age is probably the most productive viewpoint to be developed in this project concerning the conduct of life.

\section{Axioms Hi}
First, I will establish some axioms and definitions.  The self-reliance being 
discussed over the course of this paper will strictly refer to the self-reliance of 
Ralph Waldo Emerson in his 1841 essay, ``Self-Reliance," which is defined by 
Emerson to be a ``triumph of principles."  This triumph of principles, as 
Emerson argues, manifests itself in the way man carries itself in the presence 
of opposition, this opposition being identified to be one of standard social 
norms and of social propriety.  In context of the digital age, the social opposition 
resides largely in the digital world, which in this paper will refer to all forms of 
communication that occur over the screen, serving as a proxy for direct 
human-to-human interaction.  Under this definition, Facebook, Gmail, 
Twitter, and Slack are all considered digital worlds while something like the 
New York Times mobile app is not.  This is an important distinction because a 
proxy for direct human-to-human interaction requires an interchange between 
human source and human receiver.  In our New York Times mobile app 
example, the human source is fixed to be the writers of the New York Times 
while the human receiver is fixed to be the reader.  Thus, no interchange of 
roles occurs between source and receiver, thereby disqualifying the New York 
Times as a digital world.  In contrast, to participate in Facebook will 
necessarily require one to interchange between roles of receiving and sourcing 
information to friends and family, thereby making Facebook a truly digital 
world.  The coexistence of these digital worlds makes this age we live in a 
digital age.  Finally, our participation in these digital worlds makes us identify a digital self.  This digital self is different and divorced from the real self because it requires capitulating to the demands of the digital world, one that would have never occurred had our real selves made the conscious effort to not participate the digital world in the first place.  We will prove later on how the digital self not only is divorced from the real self, but also deceives the real self.  We define this deception to be one where the self strays further and further away from the ``triumph of principles" and thereby straying further away from achieving self-reliance.

\section{The Importance of Self-Reliance in the Digital Age}
The pivotal argument governing the integrity of this paper is the importance of preserving self-reliance, these ``triumphs of principles."  In this section, I will argue the merits of Emerson's defined self-reliance and why such self-reliance is necessary for how we are to spend our time living.

The best paradigm to use in illustrating the importance of self-reliance is to consider the difference between the human and the machine.  In the digital age, new technologies have arisen to give machines greater computational power, becoming more specialized and more complex in approaching the capabilities of human beings.

In such a situation, how can the human differentiate itself from the machine?
The machine is pre-conditioned, pre-programmed, predictable up to a fault.  Meanwhile, the human being, not subjected to these rules of conditioning and programmed structure, experiences a certain freedom.  As Emerson writes, if we were indeed programmed by such a God, this God has ``armed youth and puberty and manhood no less with its own piquancy and charm, and made it enviable and gracious and its claims not to be put by, if it will stand by itself."  If the human is to distinguish itself from the machine, the human must bring out the natural faculty to stand by itself.

Thus, self-reliance perpetually reinforces the human ability
to stand by itself in taking on greater liberties that a
machines is unable to afford.

If a human being wants to truly assert and demonstrates qualities of human character rather than qualities of a machine, he/she is best equipped to just exemplify the qualities of self-reliance, triumphing of principles rather than being absent of following such principles.  


\ifdraft

while the human being is not.  Any attempts t understand the human being will result in possible misconception, where human beings are isolated and divorced from the realism that unfolds.

f the human being is to truly assert his/her human qualities over.  These qualities are qualities of self-reliance

Such computational power is not necessarily divorced from human realities but

For one, self-reliance is an important assertion.  Envy is ignorance and imitation is suicide, Emerson writes.  

Consider the argument that attempts to distinguish a robot from a human being.  For one thing, the human being is great

Self-reliance is important because it addresses the inherent necessity of producing the good works of human knowledge  

It is for this self-reliance that we motivate.  Without a desire to achieve self-reliance, the goals of this paper would be aimless.
\fi
\section{The Impossibility of Emerson's Self-Reliance}

While self-reliance has been shown to be important, Emerson's self-reliance is 

Finally, by participating in these digital worlds, we taking on a new layer of identity, compromising original principles by further following more rules of the digital world and isolating ourselves from a real human identity.  The principles we follow in this digital world will at best be where nothing real is at stake to give certain principles a distinctive triumph, undeterred by qualities afflicting the everyday soul and at worst be the draining of all too many lives who have never achieved self-reliance.

\subsection{Digital Worlds Opposing Self-Reliance}
Having established some axioms and definitions, I now proceed to illustrate how the digital world exhibits the same opposition of Emerson's social world in challenging self-reliance. 

Moreover, as human beings spend more time in digital worlds, the entire human experience is confined to just a few tweets and Facebook posts, all of which can quickly emulated by such machines.  In these environments, human ingenuity loses hold of individuals

\subsection{Falsehood of Identity in the Digital-Self}
Here, I will be using Emerson's definition of self-identity.  Emerson writes that individuals who are plagued by the struggles of everyday life.  Moreover, the ideas in which I endow and ingrain within individuals like Emerson are heightened by a certain willingness to appreciate the life that is good and true.

The digital world is simply one where nothing feels particularly real.  That there is a fundamental dichotomy between what is right and what is wrong

\section{Reconciliation via Digital Minimalism}
The best solution is to compromise with ``digital minimalism".  Digital minimalism is defined to be Limiting one's access to social media for example is a great way to project notions of self-identity.  Moreover, writing about certain ideas is also another aspect of asserting self-identty.  
 Emerson's dilemna of self-reliance would be to challenge our own ideals of human understanding, of human motivation.

\section{Work}
\section{Incompatability of Emerson}

\section{The Human Connection and Empathy}

Many of these.  This goes into the classic tree description of if a tree falls, did it truly exist?  Or did it not, or are we becoming socially insignficant individuals?  How are we to behave in this day and age?  Are we running out of resources, where our time spent is not on productive work any more?  What's to say that we truly are living our lives to the fullest capacity and learning with maximum intensity?
\ifdraft

The pooling of time into interacting with this world is a pooling of life.  Thus, the interaction can be largely interpreted to reside in the minds and hearts of individuals.

Like so, social media usage has only increased in the modern world.  This is quite transparent in the world at large.

The term "digital
age" is an umbrella term that refers to a time period
where interaction with screens is made possible.  That is,
human behavior has been chiefly oriented around consensual
interaction with other human beings.
\fi

\section{Illusions of the Digital Self}
This
digital self is often based on idealization rather than
reality.  Recent studies have suggested that
human beings are spending most of their time on the
computer.

From an "anti-self-reliance" perspective, the conduct of individuals are a
social media landscape taking hold of individuals' time.

These realistic goals of individuals in the
digital age primarily consists of a social media landscape
where individuals are capitulate by partaking in social
media.  


\section{Work in the Digital Age}

\section{Deep Work and Emerson Identity}

where individuals find themselves capitulating to social media
and constructing a digital self in the modern age.  I will
examine particular dichotomies, which can be largely
interpreted as failings in Emerson's argument in
anticipating exceptions seen in our time that he might have
failed to account for during his time.

The great society that Emerson writes is one built on past
ideals of correctness and sincerity.  The life outside of
the confines and many other things leave us often utterly
convinced that there is a greater solution.

Rather than accepting that individual contributions are
often at odds with the rest of the society, the opinions
fostered by most individuals are creating something that is
more egalitarian.

The society that Emerson writes about in his Self-Reliance work is one ingrained in fundamental principles of nature. How come?  What is the natural world that Emerson writes about?

This is moreover the setup I've enjoyed.  I think it has
presented itself with many fantastic benefits.

Every Mac Desktop and Windows PC comes with a default desktop background of a natural landscape.  The world of the 21st century is completely digital, and human interaction may be part of a bygone era.  Is this a falsehood that governs our day-to-day activities or is this something more fundamental?

The idea that Emerson has contrived is one where true work is independent with the other notions of identity. 

In 'Self-Reliance', Emerson discusses the fundamental
importance of asserting the individual self, comprising of
several steps such as speaking your latent conviction, and
the like.  What is this conviction?

Moreover, individuals are rarely able to reconcile the good
and the bad.  This has been the many wonderful qualities
that have governed modern human existence.  There is finally
a question of pertinence that permeates much of human day
existence.  Why should Emerson or Nietzsche be here to
criticize or remark on these human virtues?

Evince is a great document viewer, and it's something I've
been using for all my classes.  I love to code and I love to
type out things that make me a more efficient writer.  This
is the benefit of computer science and English narration
captured in its most fundamental form.

I have often stopped to remark that these things, while good
and true often are not enough to satiate human desires,  it
would be interesting to comment on the feasibility of these
tasks at the end of the day and what qualities are the most
conducive for making such things like this possible.

The Emerson way would be to completely abandon social media
altogether.

\section{Emerson's weaknesses and addressing these weaknesses}

Is nature if ideas inspire them?  Are those ideas legitimate?  What makes us say that these ideas are part of what we want to do or say?


\section{Privileges}
Broadly speaking, Emerson's ideas also attest to a
fundamental aspect of human nature, which is our own
inconceivable understanding of humanity.

Emerson writes that, "But do your work, and I shall know you. Do your work, and you shall reinforce yourself."  But what is this work that Emerson talks about?
  
In an ideal world, everybody is able to find meaning in the
work that they do, deriving meaning from this work.
We often think about professors, CEOs, and these sorts of
individuals who are very much privileged people in the top
echelon of society, but rarely do we consider the other
half, that is, the people who are not at all positioned to think about their own lives.

However, only a certain number of people have the ability to
work towards making this work a reality, and that itself is
the infeasibility, the impossibility of Emerson's goals.

How can we best reconcile what Emerson is speaking?  How
Emerson is saying it?  It is very much our own personal
biases, judgments shrouding our ability to speak upon such
matters, and that itself is often unsatisfactory.  To
understand these large philisophical studies and 

\section{Digital Self-Reliance}

The notion of identity has become one that is intertwined with other aspects of other identity.  This identity is one that is not solely divorced from other notions of what is right or wrong.  The virtues that Emerson writes about is largely independent with that of other virtues.  Chiefly speaking, our ideas on the world and how we are to mold it into our very own creation can be seen to be tangentially related to what we describe as the optimal virtue, a world where our efforts are largely independent with that of other students.

\section{Meaningful Work in the Realm of Computational Thinking}
The computers where we think about meaningful work is often ill-contrived.  Not many people are well-maintained. 

In ``Deep Work," Newport writes about how his work is entrenched in basic principles.


\section{Nietzsche}

Romantic quest has the genius vs. them the crowd, the others.

Celestial body.  Stendhal's Red and Black is Christian because it involves the fall.  You read books, that's why you find yourself more compatible.  Then you fall in love.  Several things must happen.  There is definitely a falling away, falling from the natural habits according to your class standing, education, et cetera.  Stories that are Biblical to their core.

How that rebellion takes place, the possibility of imminent critique.  It's not until late 60's.  There is another version of this, an idealistic person comes with a simple realization in the end.  Understand the way things are run.

\section{Bound}

Spirited, Rational, Appetitive

Harmonic Development

Privatized Sublime


For Plato, his Republic is all the judgment plains.  If he can approach every situation he is in.  


\end{document}
