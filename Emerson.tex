\documentclass[12pt,letterpaper]{article}
\usepackage[utf8]{inputenc}
\usepackage[english]{babel}
\usepackage{amsmath}
\usepackage{amsfonts}
\usepackage{amssymb}
\usepackage{graphicx}
\usepackage[margin=1in]{geometry}
\usepackage{setspace}
\doublespacing
\author{Justin Zhu \\ Advised by Dr. Alexandre Victorovich Gontchar \\ Written for \textit{Conduct of Life}, taught by Professor Unger and Professor Puett}
\title{SELF-RELIANCE IN THE DIGITAL AGE}
\date{MARCH 13, 2018}

\newif\ifdraft
\pagestyle{empty}

\begin{document}

\pagestyle{empty}
\begin{titlepage}
\pagestyle{empty}
\maketitle
\pagestyle{empty}
\pagebreak
\tableofcontents
\end{titlepage}
\pagebreak
\pagestyle{plain}
\setcounter{page}{1}
\section{Thesis}
In this paper, I will reconcile Emerson's vision of
self-reliance with certain challenges of conducting life in the digital age.
This digital age has witnessed fundamental changes
in human behavior and human interaction via the rise of
digital worlds and the projection of a digital self where
this digital self is divorced, and often deceiving, of the real, true self.    The rise
of social media platforms like Facebook and messaging services like Gmail and Slack have fundamentally altered human identities, ultimately distancing human beings from the self-reliance that Emerson advocated for.  Emerson's ideas are not a solution for conduct of life in the digital age -- in fact, I will identify in subsequent sections that certain components of Emerson's self-reliance are contradicting and incompatible with such realities of everyday life.  However, being said, certain Emersonian self-reliance are still relevant and, moreover, prescriptive for living better lives in this digital age by nature of encouraging individuals to become independent thinkers.  Thus, a reconciliation of Emersonian ideas and of communication in the digital age is probably the most productive viewpoint to be developed in this project concerning the conduct of life.

\section{Axioms}
First, I will establish some axioms and definitions.  The self-reliance being 
discussed over the course of this paper will strictly refer to the self-reliance of 
Ralph Waldo Emerson in his 1841 essay, ``Self-Reliance," which is defined by 
Emerson to be a ``triumph of principles."  This triumph of principles, as 
Emerson argues, manifests itself in the way man carries itself in the presence 
of opposition, this opposition being identified to be one of standard social 
norms and of social propriety.  In the context of the digital age, the social opposition 
resides largely in the digital world, which in this paper will refer to all forms of 
communication that occur over the screen, serving as a proxy for direct 
human-to-human interaction.  Under this definition, Facebook, Gmail, 
Twitter, and Slack are all considered digital worlds while something like the 
New York Times mobile app is not.  This is an important distinction because a 
proxy for direct human-to-human interaction requires an interchange between 
human source and human receiver.  In our New York Times mobile app 
example, the human source is fixed to be the writers of the New York Times 
while the human receiver is fixed to be the reader.  Thus, no interchange of 
roles occurs between source and receiver, thereby disqualifying the New York 
Times as a digital world.  In contrast, to participate in Facebook will 
necessarily require one to interchange between roles of receiving and sourcing 
information to friends and family, thereby making Facebook a truly digital 
world.  The coexistence of these digital worlds makes this age we live in a 
digital age.  Finally, our participation in these digital worlds makes us identify a digital self.  This digital self is different and divorced from the real self because it requires capitulating to the demands of the digital world, one that would have never occurred had our real selves made the conscious effort to not participate the digital world in the first place.  The digital self not only is divorced from the real self but also deceives the real self because the individual is no longer exposing himself to the risks that come with ``triumphing of principle".  In order for a principle to triumph, there must have existed some struggle where that principle was under risk of attack.  The digital world presents us with a world where there is less risk of challenging with principles becomes nothing truly is at stake in the digital world.  With nothing truly at stake and all of the self's time invested in this digital world, the self strays further and further away from the ``triumph of principles" and thereby straying further away from achieving self-reliance in the digital world.

\section{The Importance of Self-Reliance in the Digital Age}
The pivotal argument governing the integrity of this paper is the importance of preserving self-reliance, these ``triumphs of principles."  In this section, I will argue the merits of Emerson's defined self-reliance and why such self-reliance is necessary for how we are to spend our time living.

The best paradigm to use in illustrating the importance of self-reliance is to consider the difference between the human and the machine.  In the digital age, new technologies have arisen to give machines greater computational power, becoming more specialized and more complex in approaching the capabilities of human beings.  In such a situation,  how can the human differentiate itself from the machine?  The machine is pre-conditioned, pre-programmed, predictable up to a fault.  Meanwhile, the human being, not subjected to these rules of conditioning and programmed structure, experiences certain freedoms.  As Emerson writes, if we were indeed programmed by such a God, this God has ``armed youth and puberty and manhood no less with its own piquancy and charm, and made it enviable and gracious and its claims not to be put by, if it will stand by itself."  If the human is to distinguish itself from the machine, the human must bring out the natural faculty to stand by itself.

This ability for human beings in asserting their differences from machines is necessary for conducting life in the digital age.  From an economic perspective, self-reliance enables human beings to create value that machines are unable to think of.  The machine, while more superior than the human being in performing only one task, is only capable of performing this one task.  A human being can think of new tasks and new solutions to these tasks if the human steps outside of this digital world and think deeply about his/her relationship with this digital world.  This deep thinking demands self-reliance.   Otherwise, a human being without self-reliance will be completely influenced by the digital world.  The actions of the human being without self-reliance will be indistinguishable and inferior to the machines that participate in this digital world. 

The act of thinking deeply, independently, and creatively speaks to the ``piquancy and charm" of Emerson's self-reliance.  This self-reliance is necessary for distinguishing ourselves from the machines, which is imperative in asserting human economic value in the digital world.  However, economic value is not the sole criteria we should be using to gauge the merits of self-reliance.  Deeper questions of truth and our relationships with others come from our practice of self-reliance.  Emerson describes these truthful relationships as being grounded in a faith in God, but I will demonstrate how we have to look past God in order to practice self-reliance in the digital age.

\ifdraft

In subsequent sections, I will identify deeper benefits of self-reliance in the digital world that extend beyond the merely economic and describe how the digital world is often at odds with this self-reliance we hope to achieve.


Having already identified the importance of self-reliance in the digital age, I will proceed to examine how the digital age has distanced ourselves from self-reliance, opposing these ``triumphs of principles."

even more salient are how we are to justify our lives of existence in a way that is not divorced from other human thought and human ability.  These issues plague the entirety of human ability and is moreover a central tenet for how we are to conduct our lives in the world.

Thus, self-reliance perpetually reinforces the human ability to distinguish themselves from a machine.  

machines have taken on very specialized roles in society that human beings cannot compete against.  

capable of conducting greater computations, reaching more users, and doing more grunt work than humans can.

In order to maintain a certain economic value that the machine cannot possess, the human being must be in a constant state of creation, of thinking creatively without outward control from society.  

 because digital age is witnessing more widespread automation with anthropomorphic features.  Distinguishing differences with that of a machine is also a central issue for human beings to justify their lives of existence.  If a human being is completely absorbed on the Internet day in and day out, never truly producing works of value and merely commenting on 

stand by itself, the human being in 
 a human being wants to truly assert and demonstrates qualities of human character rather than qualities of a machine, he/she is best equipped to just exemplify the qualities of self-reliance, triumphing of principles rather than being absent of following such principles.  
\fi

\ifdraft
Self-reliance is necessary for us to justify our way of life, to allow us to unleash our true potential in this world. 
As I was walking with Dr. Gontchar for dinner, he asked me what motivated me to study math and computer science.  Math was a simple answer -- it was what I loved to do ever since I was a young child.  On the other hand, I didn't have a good answer for why I was studying computer science.  As Dr. Gontchar was asking me these questions, ``What is it about programming that excites you?", I stayed mute.  Again, I didn't have an answer.

Reflecting upon this moment, I realized my decision to take computer science courses have been directly shaped by the digital world I found myself living in.  All my friends, having poured their time onto social media, have been taking classes in computer science and were applying for jobs in software engineering, and I too, was following suit.  My thoughts could not be called my own since I had never thought about why I wanted to study computer science and what computer science really meant to me other than the fact that others were studying this same thing.  I had not stepped out of the digital world, out of the tweets and the online Facebook posts to really see for myself what has been my ``latent conviction," my ``triumph of principles" in this digital age.  Now, having stepped out of this digital world, I begin to see a former hollow self who had completely capitulated to this digital world, losing all understanding of self-reliance by following others in studying computer science in order to better fit in this digital world.  To better fit in this digital world was not a good reason for studying computer science and does not demonstrate triumphs of principles.

As I scrolled through my emails and my Facebook posts, I realized the damage of the digital world was more serious than just cultivating a general aimlessness with which I treated my own life.  The digital world had forced me to project an identity that was not true to me.  Most of my Facebook posts could best characterized as what Emerson terms  ``the foolish face of praise."  Some of the praise I projected on my friends were so obsequiously distasteful, I found myself wincing.   These praise were spurred on by a digital world where everybody I know on this digital world had bestowed the same projections onto me, demonstrating the ubiquity of untruthful selves that participate in these digital worlds.

When one adopts this foolish face of praise, one loses the ability to judge critically.  The loss of good judgment speaks to what Emerson 

 As human beings spend more time in digital worlds, the entire human experience is confined to just a few tweets and Facebook posts, all of which can quickly be emulated by such machines.  In these environments, the human being loses ownership of thought, letting the mind wander aimlessly by sharing and retweeting snippets of thought left by others.


Within this digital world, Moreover, my comments were not the exception but rather the norm.  Across all digital worlds, I saw my friends posting the same flattering and empty comments that I had posted.  I had found this to be my problem where a generation have all grown to become like me.


\fi

\ifdraft
I remember taking my first introductory computer science course and how I was immediately impressed by the fact that it was the largest course at Harvard.  Such a large course appeared to make me suffer in light of other circumstances.  I had lost touch with the independent thinker who wanted to take risks and explore the unknown.  Instead, I had become a hollow, empty shell of an individual, letting loose my own 


I suppose the real reason for my decision to concentrate in computer science was so that I could briefly see 

I then thought about.

From these personal anecdotes, I extend

The social media landscape has made us extremely distracted thinkers who are incapable of thinking deeply.  Moreover, the social media landscape has turned us into individuals who are unable to experience more profound laughter.

Our identity in this world is established by how we use our time.
A machine can forward social media texts just the same, adding but a few lines of the usual platitudes found online like "lol" (a shorthand for laugh out loud), "haha" (denoting human laughter).  When we are faced with such a prospect, we find ourselves falling short of what are be essential qualities of our own personal character and personal strivings.  Many will caution against the world's own interpretations but seldom will we be able to stand atop greater worlds of intrigue. 

We can view these trends as contributing to a distancing of human beings from self-reliance.


Having established some axioms and definitions, I now proceed to illustrate how the digital world exhibits the same opposition of Emerson's social world in challenging self-reliance. 
\fi
\ifdraft

while the human being is not.  Any attempts t understand the human being will result in possible misconception, where human beings are isolated and divorced from the realism that unfolds.

f the human being is to truly assert his/her human qualities over.  These qualities are qualities of self-reliance

Such computational power is not necessarily divorced from human realities but

For one, self-reliance is an important assertion.  Envy is ignorance and imitation is suicide, Emerson writes.  

Consider the argument that attempts to distinguish a robot from a human being.  For one thing, the human being is great

Self-reliance is important because it addresses the inherent necessity of producing the good works of human knowledge  

It is for this self-reliance that we motivate.  Without a desire to achieve self-reliance, the goals of this paper would be aimless.
\fi

\section{The Impossibility of Emerson's Self-Reliance}

Emerson's depictions of self-reliance and its relationship with God do not appear consistent with Emerson's original definition of self-reliance.  Emerson writes that ``when a man lives with God, his voice shall be as sweet as the murmur of the brook and the rustle of the corn."  In this digital age, this God that Emerson describes could best be seen as a metaphor for the creators of the digital worlds.  Using Emerson's logic, if the creators of these digital worlds had a pure intention and divine purpose of enabling us to create a digital self, then we should be grateful and allow these creators to continue to absorb us in their digital worlds.

It's interesting to see how Emerson's ideas present a clear contradiction.  By trusting a higher being who is free to manipulate this digital world to his or her liking, we no longer maintain ownership and responsibility of our own actions.  Such an occurrence is the antithesis of self-reliance because self-reliance demands a ``triumph of principles" rather than an absence of principles.  Moreover, by placing complete trust in God or digital creator, we experience some of the problems that led to our loss of self-reliance in the first place.  If we were to substitute any ideas or principles that were not our own in the name of God, we are compromising our very identity and very values, capitulating to ``badges and names" and to ``large societies and institutions."  What's to say that God isn't a societal construct that is making us capitulate our own sense of identity, thereby leading us to forgo self-reliance?

Finally, there is humorous irony in noting that by adhering to Emerson's definitions of self-reliance and maintaining this relationship with God, we are following only the principles of Emerson and not principles of our own, which also makes us not truly self-reliant.

Being said, Emerson's point about God, however, is not to be fully discounted and disparaged.  When Emerson speaks of God, he is referring to certain truths that we should not alienate ourselves from, for if we were to truly live lives of principle, our principles might not be grounded in truth, which may be problematic because becoming self-reliant also implies living lives of ``goodness" as Emerson writes, and without a backing of truthful purpose in self-reliance, goodness cannot necessarily be found.
 
 In the digital world, Emerson's notion of God has changed, and our relationship with God has also changed.  Emerson's self-reliance is impossible in the digital age because of this changing relationship we have with God.  In the digital world, we are at the mercy of the creators of these digital worlds, these creators not being God but individuals and corporations whose motives might not be truthfully aligned with that of ourselves.  If we were to trust these creators of digital worlds with the same trust Emerson places in God, we are most certainly bound for complete erasure of self-reliance.  On the other hand, if we were to completely reject these digital worlds and their creators all together, we experience an undesirable solitude.  As Emerson writes, ``It is easy in the world to live after the world’s opinion; it is easy in solitude to live after our own; but the great man is he who in the midst of the crowd keeps with perfect sweetness the independence of solitude."
 
 
 
\ifdraft

When we treat the creators of the digital world with such venerance, we lose hat 


While self-reliance has been shown to be important in the digital world, Emerson's self-reliance can be seen to be overtly extreme.   

Emerson makes no mention about how 

Finally, by participating in these digital worlds, we taking on a new layer of identity, compromising original principles by further following more rules of the digital world and isolating ourselves from a real human identity.  The principles we follow in this digital world will at best be where nothing real is at stake to give certain principles a distinctive triumph, undeterred by qualities afflicting the everyday soul and at worst be the draining of all too many lives who have never achieved self-reliance.
\fi


\section{Reconciliation}
How best to live this life of the great man ``who in the midst of the crowd keeps with perfect sweetness the independence of solitude" in the digital age?  As Emerson argues, to live a life of the great man, one should not be socially isolated nor should one live after the world's opinion.   To achieve this balance between the two undesirable extremes, we must reach reconciliation between Emerson's self-reliance and the changing worlds of the digital age.

To begin this reconciliation, we start by adhering to a strict interpretation of Emerson's self-reliance and then slowly amend instances of it so that this amendment is compatible with the demands of the current digital age, to maintain socialization.  This will enable us to identify how one can better live lives of great men ``who in the midst of the crowd keeps with perfect sweetness the independence of solitude."

I also would like to just point out that there is an importance to be made between a reconciliation rather than a compromise.  A compromise suggests a mixture of two competing ideas that leave both ideas unsatisfied whereas a reconciliation suggests that one prevailing idea is still preserved amidst other competing ideas.  In our reconciliation, there is no compromise of the very idea of self-reliance.

To start, the most strict interpretation of Emerson's self-reliance would be to renounce complete usage of technology in the digital age.  The decision to not participate in digital worlds is a clear demonstration of self-reliance because it is a triumph of a clear principle to not fetter one's own life with the influences of social media and of the digital world.  However, such a renunciation also places one in a position to become completely isolated from the rest of society, isolated to the point that basic correspondence with colleagues over email, meaningful relationships developed with clients and family over phone calls, and sharing important life updates over Facebook all vanish.  In such a situation, the basic means to sustain life through work and relationships become jeopardized.  To not participate in digital worlds outright is a limitation in itself to living life freely and sustainably.

We can consider an alternative where we practice a form of digital minimalism where we impose within our lives a structured use of digital worlds, the minimal amount for us to stay connected with our friends and family, and just so that we find ourselves with time to live lives where our principles experience struggle and challenge outside of the digital world, in order for a ``triumph of principles" to materialize.  This reconciliation is one where we do not place blind faith in the creators of the digital world as Emerson did with God, but one where we place faith in our principles becoming better through having experienced real struggle.  By deliberately limiting our participation of digital worlds, we claim time again for principled solitude and henceforth achieve a renewed form of self-reliance in the digital age.

\ifdraft
However, such a strict interpretation of Emerson's self-reliance can lead to some problems.  For one, almost all communication nowadays is conducted online.

\subsection{Changing Definitions of Identity in the Digital-Self}
Here, I will be using Emerson's definition of self-identity.  Emerson writes that individuals who are plagued by the struggles of everyday life.  Moreover, the ideas in which I endow and ingrain within individuals like Emerson are heightened by a certain willingness to appreciate the life that is good and true.

The digital world is simply one where nothing feels particularly real.  That there is a fundamental dichotomy between what is right and what is wrong


My life thus far had been empty, devoid of meaning in my attempts to understand it.  In the digital age, I have found it increasingly difficult and hard to speak my convictions and learn more profoundly from my own experiences.  My fault is not a casual one, it's a distillation of many other factors, of other memories that I want to preserve.

Some reconciliations

I began tracking how much 



To best conduct life in the digital age in maintaining the 

The best solution is to compromise with ``digital minimalism".  Digital minimalism is defined to be Limiting one's access to social media for example is a great way to project notions of self-identity.  Moreover, writing about certain ideas is also another aspect of asserting self-identty.  
 Emerson's dilemna of self-reliance would be to challenge our own ideals of human understanding, of human motivation.

\section{Work}
\section{Incompatability of Emerson}

\section{The Human Connection and Empathy}

Many of these.  This goes into the classic tree description of if a tree falls, did it truly exist?  Or did it not, or are we becoming socially insignficant individuals?  How are we to behave in this day and age?  Are we running out of resources, where our time spent is not on productive work any more?  What's to say that we truly are living our lives to the fullest capacity and learning with maximum intensity?

The pooling of time into interacting with this world is a pooling of life.  Thus, the interaction can be largely interpreted to reside in the minds and hearts of individuals.

Like so, social media usage has only increased in the modern world.  This is quite transparent in the world at large.

The term "digital
age" is an umbrella term that refers to a time period
where interaction with screens is made possible.  That is,
human behavior has been chiefly oriented around consensual
interaction with other human beings.

\section{Illusions of the Digital Self}
This
digital self is often based on idealization rather than
reality.  Recent studies have suggested that
human beings are spending most of their time on the
computer.

From an "anti-self-reliance" perspective, the conduct of individuals are a
social media landscape taking hold of individuals' time.

These realistic goals of individuals in the
digital age primarily consists of a social media landscape
where individuals are capitulate by partaking in social
media.  


\section{Work in the Digital Age}

\section{Deep Work and Emerson Identity}

where individuals find themselves capitulating to social media
and constructing a digital self in the modern age.  I will
examine particular dichotomies, which can be largely
interpreted as failings in Emerson's argument in
anticipating exceptions seen in our time that he might have
failed to account for during his time.

The great society that Emerson writes is one built on past
ideals of correctness and sincerity.  The life outside of
the confines and many other things leave us often utterly
convinced that there is a greater solution.

Rather than accepting that individual contributions are
often at odds with the rest of the society, the opinions
fostered by most individuals are creating something that is
more egalitarian.

The society that Emerson writes about in his Self-Reliance work is one ingrained in fundamental principles of nature. How come?  What is the natural world that Emerson writes about?

This is moreover the setup I've enjoyed.  I think it has
presented itself with many fantastic benefits.

Every Mac Desktop and Windows PC comes with a default desktop background of a natural landscape.  The world of the 21st century is completely digital, and human interaction may be part of a bygone era.  Is this a falsehood that governs our day-to-day activities or is this something more fundamental?

The idea that Emerson has contrived is one where true work is independent with the other notions of identity. 

In 'Self-Reliance', Emerson discusses the fundamental
importance of asserting the individual self, comprising of
several steps such as speaking your latent conviction, and
the like.  What is this conviction?

Moreover, individuals are rarely able to reconcile the good
and the bad.  This has been the many wonderful qualities
that have governed modern human existence.  There is finally
a question of pertinence that permeates much of human day
existence.  Why should Emerson or Nietzsche be here to
criticize or remark on these human virtues?

Evince is a great document viewer, and it's something I've
been using for all my classes.  I love to code and I love to
type out things that make me a more efficient writer.  This
is the benefit of computer science and English narration
captured in its most fundamental form.

I have often stopped to remark that these things, while good
and true often are not enough to satiate human desires,  it
would be interesting to comment on the feasibility of these
tasks at the end of the day and what qualities are the most
conducive for making such things like this possible.

The Emerson way would be to completely abandon social media
altogether.

\section{Emerson's weaknesses and addressing these weaknesses}

Is nature if ideas inspire them?  Are those ideas legitimate?  What makes us say that these ideas are part of what we want to do or say?


\section{Privileges}
Broadly speaking, Emerson's ideas also attest to a
fundamental aspect of human nature, which is our own
inconceivable understanding of humanity.

Emerson writes that, "But do your work, and I shall know you. Do your work, and you shall reinforce yourself."  But what is this work that Emerson talks about?
  
In an ideal world, everybody is able to find meaning in the
work that they do, deriving meaning from this work.
We often think about professors, CEOs, and these sorts of
individuals who are very much privileged people in the top
echelon of society, but rarely do we consider the other
half, that is, the people who are not at all positioned to think about their own lives.

However, only a certain number of people have the ability to
work towards making this work a reality, and that itself is
the infeasibility, the impossibility of Emerson's goals.

How can we best reconcile what Emerson is speaking?  How
Emerson is saying it?  It is very much our own personal
biases, judgments shrouding our ability to speak upon such
matters, and that itself is often unsatisfactory.  To
understand these large philisophical studies and 

\section{Digital Self-Reliance}

The notion of identity has become one that is intertwined with other aspects of other identity.  This identity is one that is not solely divorced from other notions of what is right or wrong.  The virtues that Emerson writes about is largely independent with that of other virtues.  Chiefly speaking, our ideas on the world and how we are to mold it into our very own creation can be seen to be tangentially related to what we describe as the optimal virtue, a world where our efforts are largely independent with that of other students.

\section{Meaningful Work in the Realm of Computational Thinking}
The computers where we think about meaningful work is often ill-contrived.  Not many people are well-maintained. 

In ``Deep Work," Newport writes about how his work is entrenched in basic principles.


\section{Nietzsche}

Romantic quest has the genius vs. them the crowd, the others.

Celestial body.  Stendhal's Red and Black is Christian because it involves the fall.  You read books, that's why you find yourself more compatible.  Then you fall in love.  Several things must happen.  There is definitely a falling away, falling from the natural habits according to your class standing, education, et cetera.  Stories that are Biblical to their core.

How that rebellion takes place, the possibility of imminent critique.  It's not until late 60's.  There is another version of this, an idealistic person comes with a simple realization in the end.  Understand the way things are run.

\section{Bound}

Spirited, Rational, Appetitive

Harmonic Development

Privatized Sublime


For Plato, his Republic is all the judgment plains.  If he can approach every situation he is in.  
\fi

\end{document}
